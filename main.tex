\documentclass[sn-basic]{sn-jnl}
\usepackage{acro}
\usepackage{graphicx}
\usepackage{caption}
\usepackage{float} 
\usepackage{subfigure}
\usepackage{color, xcolor} 
\usepackage{soul} 
%\usepackage{subcaption}
\usepackage{caption}
\newcommand{\mathcolorbox}[2]{\colorbox{#1}{$\displaystyle #2$}}
\usepackage{amsmath,amsthm,amssymb,amsfonts}
\usepackage{array}
\usepackage{hyperref}
\makeatletter
\newcommand{\rmnum}[1]{\romannumeral #1}
\newcommand{\Rmnum}[1]{\expandafter\@slowromancap\romannumeral #1@}
\makeatother
\jyear{2023}%
\theoremstyle{thmstyleone}%
\newtheorem{theorem}{Theorem}%
\newtheorem{proposition}[theorem]{Proposition}% 
\theoremstyle{thmstyletwo}%
\newtheorem{example}{Example}%
\newtheorem{remark}{Remark}%
\usepackage{indentfirst}
\theoremstyle{thmstylethree}%
\newtheorem{definition}{Definition}%
\raggedbottom
% 其他导言区内容...

\jyear{2023}%
\theoremstyle{thmstyleone}%
% 其他设置...

% 定义缩写...
\DeclareAcronym{c}{
  short = c,
  long = Speed of light in a vacuum
}

\DeclareAcronym{h}{
  short = h,
  long = Planck constant
}

\title[ ]{Deadbeat Predictive Torque Control of Dual \\Three-Phase PMSM with Low Ripple\\ Based on Geometric Principle}
\author[1]{\fnm{Shuang} \sur{Wang}}\email{wang-shuang@shu.edu.cn}
% 其他作者...

\affil[1]{\orgdiv{School of Mechatronic Engineering and Automation}, \orgname{Shanghai University},  \city{Shanghai, \postcode{200444}, \country{China}}}

\begin{document}


\abstract{The conventional predictive torque control (PTC) suffers from the heavy torque ripple and uncertainty of the weighting factor. This article proposes a deadbeat predictive torque control (DB-PTC) scheme with low torque ripple for a dual three-phase permanent magnet synchronous motor (PMSM), in which no cost function is required. Meanwhile, a novel duration time calculation method is proposed based on the geometric principle. First, the reference voltage vector (RVV) is obtained by the principle of  the deadbeat direct torque and flux control (DB-DTFC). To avoid the computational burden, the nearest virtual voltage vector (VVV) is selected directly without the cost function. Subsequently, the torque ripple can be further reduced by calculating the corresponding action time based on the introduced geometric relationship in the volt-sec coordinate. Finally, experimental results illustrate the effectiveness of torque ripple reduction method under different working conditions.}


\keywords{Permanent magnet synchronous motor (PMSM), dual three-phase machine, predictive torque control (PTC), torque ripple, virtual voltage vector (VVV)}

\maketitle
\printacronyms[name = Abbreviations, sort = true]
\ac{c}, \ac{h}.

% 正文部分...
\section{Introduction}\label{sec1}
 
\sethlcolor {yellow} Multi-phase electric devices have been widely applied in industry and academia because of the distinct merits of high safety redundancy and low torque ripple \cite{inbook},\cite{7110356}. Particularly, among various types of multi-phase machines, dual three-phase PMSM has received extensive attention, especially in safety-critical situations, such as electric vehicles\cite{8036279}, high-speed elevators\cite{6170558}, electric ship propulsion\cite{6469209}, wind energy conversion systems\cite{7781416}, etc. For the high performance of multi-phase drives, effective control schemes are required to regulate the speed, currents and flux of the system. Compared with traditional vector control (VC) and direct torque control (DTC), the model predictive control (MPC) has become a research hotspot in the field of motor drive and power electronic control due to its inherent advantage of intuitive implementation, fast dynamic response, and flexibility to incorporate multi-objective control problems and system nonlinearities\cite{7733147}. Compared with VC, MPC obtains a better dynamic response since it does not need the current loop parameter, and directly generates the inverter drive signals\cite{5784326}. PTC can be regarded as the evolution of DTC, in which the hysteresis controller is replaced by the predictive controller while the cost function optimization takes over the switch table.Differing from the DTC, it is more accurate in voltage vector selection, which can pick out the optimal voltage vector with the minimum cost function\cite{8936566}.However, challenges exist when PTC is applied on multi-phase machines.The following issues need to be addressed:
\begin{enumerate}
\item The fifth and seventh components in the current harmonics generated by the same harmonic voltage in the dual three-phase PMSM are much larger than those in the three-phase motor, which deteriorates the steady-state current.
\item  Due to the constant magnitude and the limited quantity of the vectors, there is always an error between the desired vector and the applied one, which results in large torque ripple.
\item  A weighting factor which adjust the proportion of the stator flux and torque needs to be introduced. Yet, the determination of the weighting factor is generally by experience, which makes the optimization process both complex and time-consuming.
\end{enumerate}
 
 To address this problem 1, some effective modulation strategies are proposed in \cite{9340016}-\cite{8310594}. In \cite{9340016}, a disturbance observer is added to estimate the dead-time effect for minimizing the steady-state errors due to parameter mismatch errors. However, taking into account of the disturbance will significantly complicate the model, in addition to consider the parameter tuning of the observer. \cite{8310594} introduces a simplified PTC method where a two-step lookup table is proposed to suppress current harmonics and eliminate the variables in the \textit{x-y} subspace. However, since only one vector is adopted during each sampling axis, this method cannot effectively reduce the torque and flux position deviation. To further suppress the current harmonics, VVVs are integrated into the MPC structure to limit the injection of \textit{x-y} subspace currents with acceptable torque generation\citep{9222334,9387118,9017979}. In \cite{7946187}, voltage vectors in the \textit{$\alpha$-$\beta$} subspace and the \textit{x-y} subspace are considered together and simplified to one space, which nullifies the \textit{x-y} subspace currents. Since the amplitude of VVVs is constant, the voltage range in the \textit{$\alpha$-$\beta$} subspace is narrowed and the torque ripple would not be effectively suppressed. Based on the VVVs, the deadbeat current control is introduced in \cite{8543842} to simplify the optimization process, which has only three candidate vectors to be evaluated during each control axis, thus greatly diminishing the iterations and the computational burden. In \cite{9005390}, a robust predictive control strategy is proposed, which employs an incremental predictive model to reject the influence of rotor flux parameter in the predictions and adopts VVVs to deal with the influence of leakage inductance.

Reduced torque ripple can be obtained by incorporating duty ratio optimization\citep{6725678,9032330}. Namely, the active voltage vector selected from PTC will only be applied for a fraction of the control axis, while the rest of the time is allocated for a null vector. The duty ratio of the selected active voltage vector was calculated based on the principle of torque tracking in \cite{9032330} and the principle of optimizing both the torque and flux-linkage accuracy in\cite{8264763}, respectively. In\cite{7815310}, the amplitude of each active voltage vector is 1/3 and 2/3 of the original one, respectively, and then voltage vectors with different phases and amplitudes are synthesized. This method reduces torque ripple at the cost of a large computational burden. The concept of minimum torque ripple and stator flux regulation has been well explored for the three-phase machine drives {\cite{793368}}-{\cite{5512651}.Kang and Sul{\cite{793368}} aim to make square of the rms torque ripple during the sampling period Ts to be minimal, which increases the system complexity and computation burden.To address the problem,} the deadbeat direct torque and flux control (DB-DTFC) is utilized to reduce the number of voltage vector candidates in MPC \cite{8657983}-\cite{7893710}. With the position of the reference voltage vector (RVV) determined, the optimal in-phase virtual vectors closest to the RVV are selected as the prediction vectors, in which way avoiding testing all feasible voltage vectors, hence reducing the computation burden. In {\cite{article}}, a multi-vector-based MPTC method with low torque ripple while increasing the harmonic currents is proposed. The torque tracking accuracy is improved by using multiple candidate voltage vectors with different amplitudes. However, variation between the RVV the candidate vector still exists and the torque ripple cannot be effectively suppressed.

A tradeoff to ameliorate problem 3 is eliminating the weighting factor of the cost function. In \cite{6239590},a muti-objective optimization based on ranking criterion was utilized to evaluate torque and flux errors. The optimal voltage vector was selected by the weighted average of the error ranking values. In \cite{8307116}, \cite{8811491}, the torque characteristics are hidden by converting to single flux control. Consequently, the tedious tuning of weighting factor is no longer required in the PTC. In \cite{9275312}, the references of the voltage vector components are analytically deduced according to the torque and flux magnitude commands. Both weighting factor and cost function are eliminated in this method.

This paper proposes a PTC method for dual three-phase PMSM by using VVVs, in which no cost function is required. The vector action time is optimized by adjusting the amplitude of the candidate voltage vector based on the introduced geometric principle. This paper is organized as follows. First, the modeling of the dual three-phase PMSM drive system is presented in Section~\ref{sec2}. Section~\ref{sec3} gives a review of the PTC based on VVVs and explains its limitations. This section also establishes the volt-sec coordinate and further analyzes the causes of the toque ripple. Section~\ref{sec4} is dedicated to the proposed DB-PTC methodology and the implementation procedure. Section~\ref{sec5} provides the experimental results to verify the proposed method. Finally, the conclusions are drawn in Section~\ref{sec6}.

\section{Dual Three-Phase PMSM Drive System}\label{sec2}
\subsection {Mathematical Model}
The dual three-phase PMSM is fed by a six-phase two-level voltage source inverter (VSI), as is shown in Fig. \ref{fig1}, where the machine is composed of two sets of three-phase stator windings spatially shifted by 30 electrical degrees in two isolated neutral points. The six-phase VSI can define $2^6=64$ switching state combinations in the form of the binary number $[S_A S_B S_C]-[S_D S_E S_F]$, where $S_i$=0 or 1 (\textit{i=A, B, C, D, E, F}). Then, using the vector space decomposition (VSD) method\cite{5784326}, the components in a stationary frame can be decomposed into three orthogonal subspaces, namely $\alpha-\beta$,$x-y$ and $o_1-o_2$ subspace. The $\alpha-\beta$ subspace is related to the energy conversion process, which governs the fundamental flux and torque production, while the $x-y$ subspace is related to harmonics of 6\textit{m}±1 (\textit{m}=0,1,2,3…), which only generate losses. The $o_1-o_2$  subspace corresponds to zero-sequence harmonic components, which cannot flow due to the isolated neutral points, thus not discussed in this paper. In Fig. 2, the voltage vectors in the $\alpha-\beta$ and $x-y$ subspace are presented, and in Fig. 2(a), the $\alpha-\beta$ subspace is divided into 12 sectors, which are represented by the symbols $Q_1, Q_2$, …, $Q_{12}$.


\begin{figure}[h]%
\centering
\includegraphics[width=0.8\textwidth]{1}
\caption{Dual three-phase PMSM electric drive system.}\label{fig1}
\end{figure}





\begin{figure}[h]
	\centering
	\begin{minipage}{0.49\linewidth}
		\centering
		\includegraphics[width=0.9\linewidth]{2(a).png}
	\end{minipage}
	\begin{minipage}{0.49\linewidth}
		\centering
		\includegraphics[width=0.9\linewidth]{2(b).png}
	\end{minipage}
 \caption{Voltage vectors in the \textit{$\alpha$-$\beta$} and \textit{x-y} subspaces.}
 \label{fig_2}
\end{figure}

\begin{figure}[h]
\centering
\includegraphics[width=2.2in]{3}
\caption{The spatial distribution of VVVs.}
\label{fig_3}
\end{figure}


According to the value of amplitudes, four groups ($L_1$,$L_2$, $L_3$, and $L_4$) are formed by the 48 active vectors in the ascending order of their lengths. \hl{In this way, the voltage vectors in six phase machines can be classified in small, medium, medium large, and large voltage vectors in the $\alpha-\beta$ and $x-y$ subspaces.} The voltage vector amplitudes of~$L_1$, $L_2$, $L_3$, and $L_4$ groups are $\mid$$u_{L_1}$$\mid$ = 0.173$U_{dc}$, $\mid$$u_{L_2}$$\mid $= 0.333 $U_{dc}$, $\mid$$u_{L_3}$$\mid$ = 0.471 $U_{dc}$,$ \mid$$u_{L_4}$$\mid$ = 0.644 $U_{dc}$, where $U_{dc}$ is the DC bus voltage.

Based on the VSD coordinate transformation, the voltage, flux linkage and torque equations in the \textit{d-q}  subspace can be described as
\begin{equation}\label{eq1}
\left\{\begin{array}{l}
u_{d}=R i_{d}+d \psi_{d} / d t-\omega_{e} \psi_{q} \\
u_{q}=R i_{q}+d \psi_{q} / d t+\omega_{e} \psi_{d}
\end{array}\right.
\end{equation}

\begin{equation}
\left\{ \begin{matrix}
{\psi_{d} = L_{d}i_{d} + \psi_{f}} \\
{\psi_{q} = L_{q}i_{q}} \\
\end{matrix} \right.
\end{equation}

\begin{equation}
T_{e} = 3p_{n}\left( {\psi_{d}i_{q} - \psi_{q}i_{d}} \right)
\end{equation}



where $u_d, u_q, L_d, L_q, i_d, i_q, \psi_d, \psi_q$ are the stator voltage, inductance, current, and flux in the \textit{d-q} axis; $\omega_e$ is the rotor electrical angular speed; $\psi_f$ is the permanent magnet flux linkage; $T_e$ is the electromagnetic torque; $p_n$ is the number of pole pairs.

\subsection {Virtual Vectors}
\hl{It can be seen from Fig. {\ref{fig_2}} that the vector $u_{44}$ of  $L_{4}$ group and $u_{65}$ of $L_{3}$ group are aligned in phase in the $\alpha-\beta$ subspace and opposite in direction in the $x-y$ subspace, which means effect of these two vectors on torque and flux behavior is similar in the $\alpha-\beta$ subspace and oppositional in the $x-y$ subspace. Similarly, the other in-phase voltage vectors of the $L_{3}$ and $L_{4}$ groups share the same characteristics.
The harmonic flux linkage expression in the $x-y$ subspace is described as
}

\begin{equation}\label{fai}
\mathcolorbox{yellow}{\left\{ \begin{matrix}
   {{\psi }_{x}}={{L}_{l}}{{i}_{x}}  \\
   {{\psi }_{y}}={{L}_{l}}{{i}_{y}}  \\
\end{matrix} \right.}
\end{equation}

\hl{where $i_x, i_y, \psi_x, \psi_y$ are the current and flux linkage in the $x-y$ axis; $L_l$ is the stator leakage inductance. According to ({\ref{fai}}), the flux linkage components in the $x-y$ subspace is proportional to the harmonic current $i_x, i_y$. Thus, the voltage vectors of the $L_3$ and $L_4$ groups in the $\alpha-\beta$ subspace can be adopted to reduce the amplitude of the flux $\psi_{xy}$, thereby suppressing the harmonic currents. For the sake of brevity, the vectors synthesized by the vectors from the $L_3$ and $L_4$ groups are termed as $uu_i$ ($i$=1,2,...,12).The specific distribution is shown in Fig. {\ref{fig_3}}, which is called the $G$ group. To obtain harmonics suppression, the virtual vectors are synthesized based on the constraint that the sum of the two vectors in the $x–y$ subspace is zero. 
}




\section{PTC Based on VVVs and Its Limitations}\label{sec3}

\subsection{Method Implemention}
The control scheme of the PTC based on VVVs comprises four parts: candidate voltage vector, predictive model, cost function and the switching pulse generation.
Conventionally, the control goal is to select the candidate voltage vector from the control set (12 VVVs in the ${G}$ group and the null vector) for minimizing the torque and stator flux through a cost function. Since the harmonic current is constrained by the VVVs, no harmonic item is required in the cost function. Therefore, the cost function only sets the constraints of the torque and flux. In the real-time implementation, a computational delay caused by the digital processor will deteriorate the control performance. So, a two-step prediction method is introduced to compensate for the delay\cite{7063269}. Hence, the cost function can be defined as

\begin{equation}
g=\lvert  T_{e}^{\text{ref}}-{{T}_{\text{e}}}( k\text{+1}) \rvert+\lambda \lvert \psi _  {\text{s}}^{\text{ref}}-{{\psi }_{\text{s}}}( k\text{+1} )\rvert
\end{equation}

where "\textit{ref}" represents the reference value, and $\lambda$ represents the weighting factor.

To sum up,as is shown in Fig. \ref{fig_4}, the scheme of the PTC based on VVVs offers a simple control structure and a fast dynamic response, but it suffers some inherent shortcomings. First, all 13 candidate voltage vectors should be evaluated by the cost function, which increases the computational burden. Second, an appropriate weighting factor is required to weigh the output torque and the flux tracking accuracy. Since the lack of systematic theory, the weighting factor is generally determined by debugging in experiments.


\begin{figure}[h]
\centering
\includegraphics[width=3.8in]{4.png}
\caption{Control scheme of the PTC based on VVVs.}
\label{fig_4}
\end{figure}

\subsection{Torque Ripple Analysis}



To predict the future state of the motor, a discrete-time model of the dual three-phase PMSM is developed by using the forward Euler approximation method. Omitting the tedious derivation process, the discrete model of the PMSM in the $d-q$ frame is described as
\begin{equation}\label{eq3}
\left\{\begin{aligned}
\psi_{d}(k+2) &=\psi_{d}(k+1)+u_{d}(k+1) T_{s} \\
&+\omega_{e} \psi_{q}(k+1) T_{s}-R_{d}(k+1) T_{s} \\
\psi_{q}(k+2) &=\psi_{q}(k+1)+u_{q}(k+1) T_{s}\\
&-\omega \psi_{d}(k+1) T_{s} -R_{q}(k+1) T_{s}
\end{aligned}\right.
\end{equation}

\begin{equation}
\left\{ \begin{array}{*{35}{l}}
   \begin{aligned}
  & {{i}_{d}}\left( k+2 \right)=\frac{1}{{{L}_{d}}}[\left( {{L}_{d}}\text{-}{{T}_{\text{s}}}R \right){{i}_{d}}\left( k+1 \right)+ \\ 
 & {{L}_{q}}{{T}_{\text{s}}}{{\omega }_{\text{e}}}{{i}_{q}}\left( k+1 \right)\text{+}{{T}_{\text{s}}}{{u}_{d}}\left( k+1 \right)] \\ 
\end{aligned}  \\
   \begin{aligned}
  & {{i}_{q}}\left( k+2 \right)=\frac{1}{{{L}_{q}}}[\left( {{L}_{d}}\text{-}{{T}_{\text{s}}}R \right){{i}_{d}}\left( k+1 \right)+ \\ 
 & {{L}_{q}}{{T}_{\text{s}}}{{\omega }_{\text{e}}}{{i}_{q}}\left( k+1 \right)\text{+}{{T}_{\text{s}}}{{u}_{d}}\left( k+1 \right)] \\ 
\end{aligned}  \\
\end{array} \right.
\end{equation}

\begin{equation}
\overset{\bullet}{T_{e}} = 3p_{n}\left( {\overset{\bullet}{\psi_{d}}i_{q} + \psi_{d}\overset{\bullet}{i_{q}} - \overset{\bullet}{\psi_{q}}i_{d} - \psi_{q}\overset{\bullet}{i_{d}}} \right)
\end{equation}

According to~\cite{5316082}, the relationship between the voltage vector and the torque change at instant $k$+1 is governed by
\begin{equation}\label{eq2}
u_{q}(k + 1)T_{s} = Mu_{d}(k + 1) + B
\end{equation}

For the tested dual three-phase PMSM,  $L_d$ =  $L_q$, where:

\begin{equation}\label{1111}
M=\frac{\left(L_q-L_d\right) \psi_q(k+1)}{\left(L_d-L_q\right) \psi_d(k+1)+L_q \psi_{\mathrm{f}}}=0
\end{equation}

\begin{equation}\label{eq33}
\begin{aligned}
B &=\frac{L_{d}\left[T_{e}(k+2)-T_{e}(k+1)\right]}{3 p_{\mathrm{n}} \psi_{f}}+T_{\mathrm{s}} \omega_{e} L_{d} i_{d}(k+1)\\ 
 &+R T_{s} i_{q}(k+1)+T_{\mathrm{s}} \omega_{c} L_{d} \psi_{f}
\end{aligned}
\end{equation}


To track the torque properly, the following constraint should be satisfied

\begin{equation}
T_{e}\left( {k + 2} \right) = T_{e}^{\text{ref}}
\end{equation}


In the coordinate of $u_d$($k$+1)$T_s$-\!\! $u_q$($k$+1)$T_s$, the torque equation is represented by a line parallel to the $d$-axis. All the voltage vectors whose endpoints are located on the torque line can realize the deadbeat tracking of torque.


By (\ref{eq2}), the torque at instant $k$+2 generated by the candidate voltage vector can be described as


\begin{equation}
\begin{aligned}
&\mathcolorbox{yellow}{ T_e(k+2)=T_e(k+1)+\frac{3 p_{\mathrm{n}} \psi_f}{L_d}\left[u_q(k+1) T_{\mathrm{s}}-\right.} \\
&\left.\mathcolorbox{yellow}{ T_{\mathrm{s}} \omega_e L_d i_d(k+1)-R T_{\mathrm{s}} i_q(k+1)-T_{\mathrm{s}} \omega_e L_d \psi_f}\right]
\end{aligned}
\end{equation}

Then, the torque error can be rewritten as

\begin{equation}\label{eq345}
\Delta T_{e} = \frac{3p_{n}\psi_{f}}{L_{d}} \cdot \Delta u_{q}T_{s}
\end{equation}
where ${uu}_{i(q)}$ (${i}$=1,2,3...) is the $q$-axis component of the candidate voltage vector, ${u_q}^{ref}$ is the $q$-axis component of the RVV, ${\Delta u}_q$= ${uu}_{i(q)}$ -${u_q}^{ref}$, which represents the $q$-axis component variation between two voltage vectors. ${\Delta T}_e$ represents the torque error caused by two vectors. 

\begin{figure}[h]
\centering
\includegraphics[width=2.8in]{10}
\caption{Torque Ripple caused by the deviations between the candidate vectors and the RVV.}
\label{fig_5}
\end{figure}

It can be seen in Fig. \ref{fig_5} that the voltage vector ${uu}_2$ is located on the midline of the sector $Q_2$, which exceeds far beyond the torque line. When ${uu}_2$ is selected as the candidate vector, the phase variation between the RVV and the candidate vector ${\Delta}{\theta}$ ranges from 0° to 15°. Since the phase and amplitude of the candidate voltage vector are fixed and the RVV changes with the position of the torque line for each control cycle, there will be a large torque ripple if it is applied during the whole control axis. Meanwhile, according to \hl{{(\ref{eq345})}}, the larger the variation between the RVV and the candidate vector ${\Delta}{u}_q$ is, the larger the torque tracking error is. Therefore, to suppress the torque ripple, ${{\Delta}u}_q$ should be reduced to zero theoretically. In this paper, the action time of the voltage vector within one control axis is adjusted to realize the high-accuracy tracking of the given torque in real-time.

\section{The Proposed DB-PTC Method}\label{sec4}

\subsection{RVV Calculation}
\begin{figure}[h]
\centering
\includegraphics[width=3.8in]{5}
\caption{Control scheme of the proposed DB-PTC.}
\label{fig_6}
\end{figure}


The control scheme of the proposed DB-PTC method is illustrated in Fig.\ref{fig_6}. The procedure of the proposed method includes: first, calculating the RVV based on the DB-DTFC; then selecting one VVV from the candidate vectors according to the position of the RVV; finally adjusting the action time of the candidate voltage vector. The specific implementation of the proposed DB-PTC is elaborated as follows.


Apart from the torque command, the stator flux should also be regulated. The flux components at instant $k$+2 are calculated as in (\ref{eq3}). Considering that the magnitude of the the resistance term and cross-coupling term is very small{\cite{5766036}}, the flux components are approximated as
\begin{equation}\label{eq1}
\left\{ \begin{matrix}
{\psi_{d}\left( {k + 2} \right) = \psi_{d}\left( {k + 1} \right) + u_{d}\left( {k + 1} \right)T_{s}} \\
{\psi_{q}\left( {k + 2} \right) = \psi_{q}\left( {k + 1} \right) + u_{q}\left( {k + 1} \right)T_{s}} \\
\end{matrix} \right.
\end{equation}


Subsequently, the magnitude of the stator flux linkage $\psi_s$ at instant $k$+2 can be expressed as


\begin{equation}\label{eq4}
\begin{array}{c}
\psi_{s}^{2}\left( {k + 2} \right) = \psi_{d}^{2}\left( {k + 2} \right) + \psi_{q}^{2}\left( {k + 2} \right)\\
 \text{=}{{\left[ \psi_{d}^{{}}\left( k+1 \right)+u_{d}^{{}}\left( k+1 \right){{T}_{\text{s}}} \right]}^{2}}+{{\left[ \psi _{q}^{{}}\left( k+1 \right)+u_{q}^{{}}\left( k+1 \right){{T}_{\text{s}}} \right]}^{2}} 
 \end{array}
\end{equation}



As is shown in Fig. \ref{fig_7}, the stator flux equation (\ref{eq4}) is represented by a circle geometrically in the coordinate of $u_d$($k$+1)$T_s$ - $u_q$($k$+1)$T_s$ \cite{7815310}. Any vector whose endpoint falls on this stator flux linkage circle will generate the requested stator flux linkage magnitude by the next sample time, thus achieving deadbeat flux control. The intersection point of the torque line and the stator flux circle is the endpoint of RVV to track the torque and flux accurately, which makes the actual flux and torque equal to the reference one. Therefore, the voltage vector can be calculated by
\begin{equation}\label{34}
\begin{array}{c}
T_{s}^{2} u_{d}^{2}(k+1)+2 \psi_{d}(k+1) T_{s} u_{d}(k+1)  \\
+\left[\psi_{d}^{2}(k+1)+\left(\psi_{q}(k+1)+B\right)^{2}-\left(\psi_{s}^{\mathrm{ref}}\right)^{2}\right]=0
\end{array}
\end{equation}

Substituting {(\ref{1111})} into {(\ref{eq2}) and from {(\ref{34})}}, the reference voltage in the \textit{d-q} subspace can be expressed as


\begin{equation}\label{eq0}
\left\{ \begin{array}{*{35}{l}}
   {{u}_{d}}\left( k+1 \right)=\frac{-b\pm \sqrt{{{b}^{2}}-4ac}}{2a}  \\
   {{u}_{q}}\left( k+1 \right)=\frac{B}{{{T}_{\text{s}}}}  \\
\end{array} \right.
\end{equation}
Where

\begin{equation}
\left\{ \begin{array}{*{35}{l}}
   a=T_{\text{s}}^{2}  \\
   b=2{{\psi }_{d}}\left( k+1 \right){{T}_{\text{s}}}  \\
   c=\psi _{d}^{2}\left( k+1 \right)+{{\left[ {{\psi }_{q}}\left( k+1 \right)+B \right]}^{2}}-{{\left( \psi _{\text{s}}^{\text{ref}} \right)}^{2}}  \\
\end{array} \right.
\end{equation}
 

 Between the two voltage vectors obtained by {(\ref{eq0})}, $u_d$ with a smaller amplitude is selected due to the constraint of DC bus voltage.

The phase angle of the RVV in the \textit{$\alpha$-$\beta$} subspace can be obtained by

\begin{equation}\label{eq5}
\theta^{ref}\left( {k + 1} \right) = {\mathit{\arctan}\frac{u_{\beta}\left( {k + 1} \right)}{u_{\alpha}\left( {k + 1} \right)}}
\end{equation}

\begin{figure}[h]
\centering
\includegraphics[width=2.5in]{7}
\caption{Principle of Vector Action Time Calculation.}
\label{fig_7}
\end{figure}

\subsection{Action Time Calculation}

 Considering that the amplitude of VVVs in the \textit{x-y} subspace is zero, which can theoretically eliminate the \textit{x-y} subspace current and suppress the current harmonics, the VVVs of the $G$ group are introduced as candidate voltage vectors in the proposed DB-PTC method. As is shown in Fig. \ref{fig_3}, the candidate voltage vector should be selected from the sector in which the RVV is located. The phase angle calculated from (\ref{eq5}) determines which sector the RVV lies in, and then the virtual vector in the same sector will be selected directly. For example, if RVV locates in sector $Q_2$, only ${{uu}_2}$ is selected as the candidate voltage vector without cost function, which simplifies the prediction model and reduces the computational burden. Furthermore, the amplitude of the output voltage vector can be changed by adjusting the action time of the candidate voltage vector, thus suppressing the torque ripple. Therefore, the optimization of the vector action time can be equivalent to a geometric problem, which means calculating the optimal action time according to the position of the torque line. 
 
Fig. \ref{fig_7} shows the principle of the action time calculation. From {(\ref{eq2})}-{(\ref{eq33})}, the equation of torque line in volt-sec coordinate can be expressed as

\begin{equation}\label{43}
\\
   {{u}_{q}}(k+1){{T}_{\text{s}}}\text{=}\frac{{{L}_{d}}\Delta {{T}_{\text{e}}}(k+1)}{3{{p}_{\text{n}}}{{\psi }_{\text{f}}}}+{{T}_{\text{s}}}{{\omega }_{\text{e}}}{{\psi }_{d}}(k+1)+\frac{R{{T}_{\text{s}}}{{\psi }_{q}}(k+1)}{L_{d}^{{}}} \\ 
\end{equation}

From {(\ref{43})}, the torque line is parallel to the ${d}$-axis coordinate of the volt-sec coordinate, and intersects at the point (0, B). In geometry, the modulus of the candidate vector changes in accord with its action time, thus adjusting the amplitude. To realize the deadbeat tracking of the torque, the endpoint of the vector $u_{opt}t_{out}$ should be located on the torque line. Therefore, the following relationship can be obtained by a simple triangle relation:

\begin{equation}
t_{\mathrm{out}}=\frac{B}{\boldsymbol{u}_{\mathrm{opt}} \cdot \sin \left(\theta_{u}\right)}
\end{equation}
where $u_{\mathrm{opt}}$ indicates the candidate voltage vector and $t_{\mathrm{opt}}$ is its action time; $\theta_{u}$ indicates the angle between the candidate voltage vector and the ${d}$-axis of the volt-sec coordinate.

\hl{Meanwhile, the action time can be calculated by projecting $u_{ref}$$T_s$ onto the direction of the candidate voltage vector, thus obtaining the minimum variation between the RVV and the candidate voltage vector{\cite{7892874}}. }
\subsection{Switching Pulse Generation}
Once the candidate vector and the action time of the vector are determined, the switching pulse can be generated through the seven-segment SVPWM modulation. Specifically, two situations will be discussed.



\begin{figure}[h]
\centering
\includegraphics[width=3.5in]{8}
\caption{Switching pulses generation for virtual vectors. (a)\textit{$uu_1$} (b) \textit{$uu_2$} (c) \textit{$uu_{2}^{\prime}$ }       }
\label{fig_8}
\end{figure}
For the $G$ group, \hl{zero average value of the $x–y$ subspace is obtained by using two effective voltage vectors in each sample time.} Besides, the switching pulses for some of the virtual vectors are not standard PWM pulses in one axis, which makes the implementation difficult. According to their different switching pulse generation features, the 12 virtual vectors can be divided into two groups, namely ${{G}_1}$(${{uu}_1}$, ${{uu}_3}$, ${{uu}_5}$, ${{uu}_7}$, ${{uu}_9}$, $uu_{11}$) and ${{G}_2}$(${{uu}_2}$, ${{uu}_4}$, ${{uu}_6}$, ${{uu}_8}$, ${{uu}_{10}}$, ${{uu}_{12}}$).


\hl{The action time of the effective voltage vectors is calculated by inserting null vectors, achieving the adjustment of the amplitudes within each control cycle.
Considering the amplitudes of the reference voltage vector and the candidate voltage vector, the vector action time $t_{{opt}{G_i}}$ is calculated as follows:}

\begin{equation}
\begin{aligned}
\mathcolorbox{yellow}{ t_{\mathrm{opt} {G_i}}}\mathcolorbox{yellow}{=\frac{\left\vert{u_{\mathrm{ref}}}\right\vert}{\left\vert{{uu}_{\mathrm{G_i}}}\right\vert }T_{\mathrm{s}}}
\end{aligned}
\end{equation}

\hl{Set the action time of the null vectors ($u_{77}$,$u_{00}$) as $t_0$:}
\begin{equation}
\mathcolorbox{yellow}{t_{0} = T_{s}-t_{opt{G_i}}}
\label{---}
\end{equation}

\hl{Considering the dead time and winding phase current sampling characteristics in the actual system, this paper sets the total action time of null vectors as $t_0$=0.2$T_s$ and the action time of the output voltage vector as $t_{{opt}{G_i}}$ =0.8$T_s$.}
\begin{enumerate}

    \item ${{G}_1}$ group: Adjust the action sequence of the effective voltage vector in one control cycle, and then insert null vectors to standardize the switching pulse sequence of the VVVs in group ${{G}_1}$. As shown in Fig. \ref{fig_8}(a), it is the standard PWM switching sequence of ${{uu}_1}$. \hl{Let the action time of two vectors from $L_4$ and $L_3$ groups be $t_1$ and $t_2$, respectively, and the following constraints should be satisfied:}



\begin{equation}
\mathcolorbox{yellow}{\left\{ \begin{matrix}
  {{t}_{1}}+{{t}_{2}}={{t}_{\mathrm{optG_1}}} \\
  0.173{{\text{U}}_{\text{dc}}}\cdot {{t}_{1}}-0.471{{\text{U}}_{\text{dc}}}\cdot {{t}_{2}}=0  \\
\end{matrix} \right.}
\end{equation}
\hl{So:}

\begin{equation}
\mathcolorbox{yellow}{\left\{ \begin{matrix}
 {{t}_{1}}=0.731{{t}_{\mathrm{optG_1}}} \\
 {{t}_{2}}=0.269{{t}_{\mathrm{optG_1}}} \\
\end{matrix} \right.}
\end{equation}
\hl{The magnitude of the VVV in the $x-y$ subspace is zero, and the magnitude in the $\alpha-\beta$ subspace can be calculated by}

\begin{equation}
\begin{aligned}
\mathcolorbox{yellow}{\left\vert{u}{u}_{G_1} \right\vert}\mathcolorbox{yellow}{=\frac{0.731 {t}_{\mathrm{optG_1}} \cdot 0.644 \mathrm{U}_{\mathrm{dc}}+0.269 {t}_{\mathrm{optG_1}} \cdot 0.471 \mathrm{U}_{\mathrm{dc}}}{T_{\mathrm{s}}} }
& \mathcolorbox{yellow}{=0.597 \mathrm{U}_{\mathrm{dc}}}
\end{aligned}
\end{equation}

    \item ${{G}_2}$ group: Fig. \ref{fig_8}(b) is the switching sequence generated by ${{uu}_2}$ from ${{G}_2}$ group. It can be seen that no matter how the action sequence of the voltage vectors changes, the standard PWM pulse cannot be generated. Similarly, the switching pulse sequences of the other vectors from ${{G}_2}$ group have similar problems. The method adopted in this paper is to replace the voltage vectors from ${L_4}$ group with two voltage vectors from ${L}_{2}$ group \cite{8486708}. For instance, for virtual vector ${{uu}_2}$, the actual vector ${u}_{64}$ can be replaced by two actual vectors ${u}_{04}$ and ${u}_{67}$, since the composition of ${u}_{04}$ and ${u}_{67}$ is equivalent to ${u}_{64}$ both in \textit{$\alpha$-$\beta$} and \textit{x-y} subspaces. Therefore, the virtual vector synthesized by ${u}_{46}$, ${u}_{04}$ and ${u}_{67}$ (termed as ${{uu'}_2}$) is equivalent to the virtual vector ${{uu}_2}$. Nevertheless, the virtual vector ${{uu'}_2}$ can present a standard PWM switching sequence and is easy to be implemented, as shown in Fig. \ref{fig_8}(c). For the ${{G}_2}$ group, the action time of the vectors from Group ${L}_{2}$ is ${t}_3$ and that of the vectors from Group ${L}_3$ is ${t}_4$.  Meanwhile, the following constraints should be satisfied:
    

\begin{equation}
\left\{ \begin{matrix}
{2t_{3} + t_{4} = \mathcolorbox{yellow}{{t}_{\mathrm{optG_2}}}} \\
{0.173U_{\text{dc}} \cdot t _{3} - 0.471U_{\text{dc}} \cdot t_{4} = 0} \\
\end{matrix} \right.
\end{equation}
 So:
\end{enumerate}


\begin{equation}
\left\{ \begin{matrix}
{t_{3} = \mathcolorbox{yellow}{0.4223{t}_{\mathrm{optG_2}}}} \\
{t_{4} = \mathcolorbox{yellow}{0.1554{t}_{\mathrm{optG_2}}}}\\
\end{matrix} \right.
\end{equation}




\section{Experimental Verification}\label{sec5}
\begin{figure}[h]
\centering
\includegraphics[width=3in]{testttt.png}
\caption{\hl{The test bench.  } }
\label{fig_9}
\end{figure}
As is shown in Fig. {\ref{fig_9}}, a dual three-phase PMSM test bench is established to implement the proposed DB-PTC.\hl{The load consists of a servo motor, and the two motors are operated in a cascaded mode.}  The motor is driven by a two-level voltage source inverter with a sampling frequency of 5 kHz. \hl{The control actions are performed using TC264 from Infineon.}The parameters of the experimental motor are shown in Table I.

\hl{The conventional method mentioned in section~{\ref{sec3}}, which directly suppresses the torque ripple through a cost function, is termed PTC1. The method optimizing the action time of the voltage vector by minimizing the variation between the RVV and the candidate voltage vector in {\cite{7892874}} is termed PTC2, while the proposed method letting the endpoint of the vector located on the torque line is termed PTC3. It is noted that the proposed method as well as the two comparison methods select the candidate voltage
vector from the 12 VVVs.} 


 \begin{figure}[h]
\centering
\includegraphics[width=3.8in]{revisedFIG10.png}
\caption{\hl{Steady state performance at 100 r/min with 3 N$\cdot$m load. (a) PTC1. (b) PTC2. (c) PTC3.}}
\label{fig_10}
\end{figure}



\begin{figure}[h]
\centering
\includegraphics[width=4.2in]{revisedfig11.png}
\caption{\hl{Steady state performance at 100 r/min with 5 N$\cdot$m load. (a) PTC1. (b) PTC2. (c) PTC3.
  }}
\label{fig_11}
\end{figure}


 \hl{First, the experimental results at the steady-state under 100r/min with a 3 N$\cdot$m are presented in Fig. {\ref{fig_10}}. From top to bottom, the waveforms given in Fig. {\ref{fig_10}} are stator phase currents, harmonic currents in the ${z_1-z_2}$ subspace and torque. Besides, flux reference tracking experiments are added under the rated torque. The flux tracks the reference value well and features low ripple in three methods by adopting the principle of DB-DTFC.} It can be seen from Fig. {\ref{fig_10}} and Fig. {\ref{fig_11}} that the phase currents of the three methods are basically sinusoidal waveforms. A highly distorted phase current can be observed in Fig. {\ref{fig_10}}(a) for PTC1, while the phase current waveforms of the PTC2 and PTC3 are relatively smooth and the current ripple is smaller. Since the VVVs can achieve zero harmonic voltage theoretically and are applied in three methods, it can be justified that the harmonic currents are kept small and suppressed effectively. \hl{In addition, from the results of Fig. {\ref{fig_10}} and Fig. {\ref{fig_11}}, it can be seen that PTC2 and PTC3 can achieve much better torque tracking accuracy, which proves the superiority of these two methods in suppressing the torque ripple compared with the PTC only based on VVVs. 
 Moreover, PTC3 achieves lowest level of torque ripple due to the direct torque tracking method based on geometric principle.}
\begin{figure}[h]
\centering
\includegraphics[width=4.2in]{revisedfig12.png}
\caption{\hl{Dynamic response with load suddenly changes at 100 r/min. (a) PTC1. (b) PTC2.(c) PTC3.}}
\label{fig_12}
\end{figure}


\begin{table}[h]
    \centering
  \caption{Parameters of Experimental Motor}
    \begin{tabular}{cc}
    \midrule
  
    Specification & Value \\
    \midrule
    Rated torque & 5 N·m \\
    Rated current & 60 A \\
    Stator resistance & 0.0225$\Omega$ \\
    $d$-axis inductance & 0.053 mH \\
    $q$-axis inductance & 0.053 mH \\
    Number of pole pairs & \multicolumn{1}{c}{5} \\
    Stator flux & 0.0056 Wb \\
    Stator leakage inductance & 0.0027 mH \\
   \midrule
 
    \end{tabular}%
  \label{tab:addlabel}%
\end{table}%



\begin{table}[h]
    \centering
  \caption{Torque Ripple at 100r/min with 3 N·m}
    \begin{tabular}{ccc}
    \midrule
    \multicolumn{1}{c}{} & ${{T_e}^{ref}}$ & \multicolumn{1}{p{6em}}{${{T_e}^{ripple}}$/${{T_e}^{ref}}$} \\
    \midrule
    PTC1 & 0.4235 N·m & 14.12\% \\
   \hl{ PTC2} & \hl{0.1209 N$\cdot$m }& \hl{4.03\% }\\
    PTC3 & 0.0329 N·m & 1.10\% \\
     \midrule
    \end{tabular}%
  \label{tab:addlabel}%
\end{table}%

\begin{table}[h]
  \centering
  \caption{Torque Ripple at 100r/min with 5 N·m}
     \begin{tabular}{ccc}
    \midrule
    \multicolumn{1}{c}{} & ${{T_e}^{ref}}$ & \multicolumn{1}{p{6em}}{${{T_e}^{ripple}}$/${{T_e}^{ref}}$} \\
    \midrule
    PTC1 & 0.3103 N·m & 6.21\% \\
    \hl{PTC2} & \hl{0.1527 N$\cdot$m }& \hl{3.10\%} \\
    PTC3 & 0.0533 N·m & 1.07\% \\
     \midrule
    \end{tabular}%
  \label{tab:addlabel1}%
\end{table}%

To compare the steady-state performance of each control method more accurately, the standard version is introduced to compare the torque ripple, which can be formulated as

\begin{equation}
T_{e}^{\text{ripple}} = \sqrt{\frac{1}{n}{\sum_{i = 1}^{n}\left( {T_{e} - T_{e}^{\text{ref}}} \right)^{2}}}
\end{equation}
where ${n}$ is the number of the samples.

It can be verified in Table {\ref{tab:addlabel}} and Table {\ref{tab:addlabel1} }that the torque ripple of PTC3 is significantly smaller than PTC1 and PTC 2 under the condition of both 3 N·m and 5 N·m reference torque values. These experimental results demonstrate that the proposed method has obvious improvement in steady-state ripple and current quality by optimizing the action time of the applied vector.


Apart from the steady-state performance comparison, the dynamic response of two control schemes is also investigated. The sudden load change is carried out and the transient state waveforms are given in Fig. {\ref{fig_12}}, including the phase current, torque, flux, and partially enlarged view of the torque. The load torque (3 N·m) is suddenly applied to the motor, and then it changes back to 0 N·m in a sudden instant. It can be observed from the partially enlarged view that the torque tracking time is kept within 100ms, thus the torque command is tracked satisfactorily in the three methods.

To sum up, the effectiveness of the proposed method is confirmed in the steady-state performance improvement by reducing the torque ripple and enhancing the current quality. In the meantime, the fast dynamic response is also maintained.

\section{Conclusion}\label{sec6}
In this paper, the PTC method based on VVVs is firstly analyzed, and it is found that the candidate vectors with a fixed amplitude will result in a large torque ripple. Then, a PTC method with low torque ripple is proposed in this paper. The amplitude of the candidate voltage vector is adjusted by optimizing the action time of the vector. First, the RVV is obtained by the principle of DB-DTFC. Then, according to the position of the RVV, the virtual vector located in the same sector will be selected as the candidate voltage vector directly without the cost function, hence simplifying the model complexity. Subsequently, the action time of the candidate voltage vector is calculated by letting the endpoint of the vector located on the torque line in the volt-sec coordinate, which realizes the high-accuracy tracking of the torque. In the meantime, since the position of the output voltage vector can be optimally tracked with the given torque value in real-time, the characteristic of  the fast dynamic response of conventional PTC can be inherited. Better phase current performance with a much lower torque ripple and a maintained dynamic performance are observed in the experimental results.
%\bibliographystyle{sn-basic}
\section*{Declarations}

\begin{itemize}
\item Funding

This study is supported by Shanghai Natural Science Foundation under Grant 19ZR1418600.

\item Conflict of interest/Competing interests 

The authors declare that they have no conflicts of interest to this work.

\item Ethics approval 

Not applicable

\item Availability of data and materials

Not applicable

\item Authors' contributions

Author 1(Shuang Wang): Methodology, Writing - Review and Editing, Simulink/Experiment verification and guidance, Supervision, Project administration;

Author 2(Ying Zhang): Research investigation, Simulink/Experiment verification, Writing - Original Draft preparation;

Author 3(Jianfei Zhao):Research support, Funding acquisition;

Author 4(Deliang Wu):Writing - Review and Editing

All authors read and approved the final manuscript.
\end{itemize}

\bibliography{bibliography}
\end{document}
